\chapter{Progetto}
\label{ch:job_motivation}
Il progetto sviluppato durante il tirocinio presso l'Universtià degli studi di Torino, con la supervisione del professore Michele Garetto, si pone come obiettivo 
quello di creare una cassetta delle lettere smart, la quale ogni volta in cui viene ricevuta posta è in grado di avvertire e far interagire l'utente 
grazie all'invio di e-mail a un specifico indirizzo preimpostato.

Per fare ciò, sono stati utilizzati diversi servizi illustrati nel dettaglio nel seguente paragrafo.


\section{Soluzione Proposta}
\label{sec:solution}
La soluzione proposta in questa tesi utilizza una ``classica" cassetta delle lettere resa smart grazie all'integrazione di diversi sensori e servizi alla scheda 
Raspberry Pi 3 Model B, di seguito verrà illustrato il suo funzionamento.

Il soggetto che invia la posta dovrà semplicemente inserire la lettera nel bucalettere, a questo punto vengono innescati una serie di eventi:
\begin{itemize}
    \item Grazie al sensore di prossimità Qwiic Proximity Sensor (VCNL4040), in Figura~\ref{photo_sensor}, viene rilevato il passaggio di un'oggetto, la lettera nel 
    nostro caso, quindi viene scattata una foto con la Raspberry Pi Camera~\ref{photo_camera}.
    \item Quindi vi è l'interazione con alcuni servizi di AWS(Amazon Web Services), in particolare, la foto appena scattata viene caricata in Amazon S3, questo evento
    lancia la funzione Lambda la quale verifica se le foto caricata è un volto riconosciuto da Amazon Rekognition e quindi presente in Amazon Dynamodb.
    \item A questo punto possono verificarsi solo due casi:
    \begin{itemize}
        \item Il volto è stato riconosciuto, quindi viene inviata una e-mail che notifica la ricezione di nuova posta con il nome del mittente.
        \item Il volto non è stato riconosciuto, quindi viene inviata un'e-mail che notifica la ricezione di nuova posta con allegata
        la foto del mittente.
    \end{itemize}
    \item Infine l'utente potrà insierire i volti non riconosciuti da Amazon Rekognition nei database inoltrando l'email ricevuta, inserendo nel corpo dell'e-mail 
    la parola chiave ADD seguita da nome e cognome con cui si vuole memorizzare il volto. Quindi dopo gli opportuni controlli, i volti inseriti saranno pronti 
    ad un prossimo eventuale controllo ad essere riconosciuti.
\end{itemize} 

Tutto il codice e funzionamento nel dettaglio del progetto verrano illustrati nel capitolo ~\ref{ch:development}.