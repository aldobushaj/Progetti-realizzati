\chapter{Conclusioni}
\label{ch:conclusions}
In questa tesi si è cercato di rielaborare il concetto di cassetta delle lettere classico, proponendo un sistema che offrisse nuovi servizi e che fosse sviluppato 
usando diverse tecnologie e dispositivi all'avanguardia, il suo sviluppo mi ha permesso inoltre di approfondire diversi argomenti, terminando con lo sviluppo di un 
progetto che li comprendesse tutti: 
\begin{itemize}
    \item Programmazione sul dispositivo Raspberry Pi.
    \item Gestione in remoto di sistemi informatici.
    \item Interazione con i servizi messi a dispozione da Amazon Web Services.
    \item Programmazione in Python.
\end{itemize}
Un concetto chiave di questo percorso è sicuramente Internet of Things, la domotica in particolare, occupa un ruolo chiave su questo nuovo modo di vedere Internet. 

In futuro, sempre più accessori e dispositivi saranno progettati per essere connessi alla rete, come si è visto nel Capitolo~\ref{ch:introduzione} infatti i dispositivi 
connessi sono in notevole aumento anno dopo anno, tutto questo naturalmente a vantaggio degli utenti, i quali potranno godere dei benefici di questa rivoluzione. Infatti, 
questa connessione (e interconnessione) dei dispositivi consente non solo di aggiungere funzionalità alla nostra abitazione ma anche di evitare sprechi di cibo e di 
elettricità.

Il progetto sviluppato si propone di essere un prototipo di un sistema di domotica, ha una struttura modulare e le funzionalità sono facilmente estensibili, infatti, 
nonostante in questo progetto sia stato sviluppato solo il tema del controllo a distanza della cassetta delle lettere, con l'aggiunta di alcuni sensori, si può 
aumentare la sicurezza dell'abitazione o aumentarne le funzionalità. 

Ad esempio, è possibile inserire ulteriori sensori, che gestiscano situazioni critiche come fughe di gas, incendi e allagamenti
all'interno dell'abitazione, o semplicemente rendere possibile il controllo a distanza dei dispositivi casalinghi come frigoriferi, luci o ancora, riscaldare la casa 
poco prima che si rientri o far abbassare le serrande automaticamente una volta rilevati i raggi solari.
\hbox{}
\thispagestyle{empty}
\newpage 