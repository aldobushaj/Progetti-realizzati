\chapter{Abstract}
\label{ch:abstract}
Oggigiorno tra ufficio, hobby e impegni vari trascorriamo sempre meno tempo a casa, quindi a volte, se si aspetta una lettera importante o semplicemente per
comodità, l'deale sarebbe che qualcuno o meglio qualcosa avvisi l'utente (il proprietario della cassetta delle lettere) ogni qualvolta viene ricevuta nuova 
posta informandolo eventualmene anche dell'identità del mittente.

Usando il prototipo sviluppato, quando il mittente inserisce una lettera nel bucalettere, una fotocamera scatta una foto che verrà caricata nel database. A questo
punto, tramite un'email all'indirizzo precedentemente impostato, l'utente viene notificato dell'avvenuta ricezione di nuova posta indicandone anche l'identità.

Il progetto è stato implementato su una scheda Raspberry-Pi, mentre l'applicazione, è stata scritta usando il linguaggio Python.
Sono stati utilizzati diversi servizi di AWS (Amazon Web Services) sia per quanto riguarda il riconoscimento facciale che per il caricamento delle foto scattate nel 
database, i più importanti sono:
\begin{itemize}
    \item Amazon Rekognition
    \item Amazon Simple Storage Service (Amazon S3)
    \item Amazon DynamoDB
    \item AWS Lambda 
\end{itemize} 