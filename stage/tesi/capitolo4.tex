\chapter{Sviluppo}
\label{ch:development}

In questo capitolo verranno illustrate tutte le fasi riguardanti lo sviluppo del progetto e quali risultati sono stati ottenuti per quanto riguarda il prodotto finale. 

In particolare nella Sezione~\ref{sec:product_development} si vedrà dettagliatamente il sistema che è stato sviluppato, di quali parti si compone
e come queste ultime interagiscono tra di loro. Nella Sezione~\ref{sec:phases_of_development} invece, si vedrà come è stato raggiunto il risultato finale.

\section{Prodotto sviluppato}
\label{sec:product_development}

Il file principale che mette in funzione tutti i componenti(sensore, fotocamera e servizi AWS) è \textbf{``project.py''}, il quale contiene uno script in linguaggio 
Python da far girare sulla scheda Raspberry Pi.
Dopo aver importato le opportune librerie e collegato i vari componenti alla scheda Raspberry Pi, creato i vari database e la funzione Lambda dalla console AWS
(configurazioni hardware e software verrano tratti nel dettaglio nella Sezione~\ref{sec:phases_of_development}) si è scritto come riportato nel 
Listato~\ref{ciclo_while} la parte di codice che gestisce la parte principale del progetto.

\begin{lstlisting}[language=Python,frame=single,caption=Codice Python ,captionpos=t,label=ciclo_while]
    i2c = busio.I2C(board.SCL, board.SDA)
    sensor = adafruit_vcnl4040.VCNL4040(i2c)
    #variabile rappresentante il tempo trascorso
    contatore = 0  
    while True:
        #rilevato il passaggio della lettera
        if sensor.proximity >= 10: 
            print( ``Proximity:'', 
            sensor.proximity , ``then take a photo'' )
            capture_image() #scatto la foto
            upload_image()  #la carico in S3
            send_email()    #e mando l'e-mail
            sleep(3)
        else:
            time.sleep( 0.5 )
            contatore = contatore+1
            # passato 1 minuto procedo
            if( contatore == 120 ):
                #leggendo le nuove e-mail ricevute
                read_email_from_gmail()
                #azzero nuovamente il contatore
                contatore=0     
\end{lstlisting}
La variabile ``\texttt{contatore}'' si occupa di tener conto del tempo trascorso, una volta inizializzata la variabile ``\texttt{sensor}'' si entra nel ciclo while 
infinito il quale, rilevato il passaggio di un oggetto (la lettera nel nostro caso) gestito nell'If come illustrato nel Listato~\ref{ciclo_while}, viene eseguito il 
metodo \textit{capture\_image()}, Listato~\ref{capture_image}, dove verrà scattata una foto solo dopo aver effettuato una sleep
\footnote{https://www.programiz.com/python-programming/time/sleep} di 2 secondi in modo da dare il tempo ai sensori di luce della fotocamera di stabilizzarsi, 
dopodiché, si potrà procedere memorizzando temporaneamente la foto appena scattata nel Desktop sulla scheda Raspberry Pi.
\begin{lstlisting}[language=Python,frame=single,caption=Funzione capture\_image,captionpos=t,label=capture_image]
    def capture_image():
        camera = PiCamera()
        camera.start_preview()
        sleep(2)
        camera.capture( `/home/pi/Desktop/' + filename )
        camera.stop_preview()
        camera.close()
        return    
\end{lstlisting}
A questo punto si può procedere caricando la foto appena scattata su AWS, nel database Amazon S3, all'interno del bucket ``images-sender'' creato precedentemente, come vedremo 
più avanti, il codice utilizzato per caricare la foto su Amazon S3 viene illustrato nel Listato~\ref{upload_image}.
\begin{lstlisting}[language=Python,frame=single,caption=Funzione upload\_image code,captionpos=t,label=upload_image]
    def upload_image():
        file = open( filename, `rb' )
        ret = s3.put_object( Bucket = `images-sender' , Key=filename , Body=file )
        return
\end{lstlisting}
Il prossimo passo sarà quello di inviare una e-mail che notifica la ricezione di nuova posta all'utente, per far ciò viene eseguito il metodo \textit{send\_email()}.

Inizialmente si aspetta che venga eseguita la funzione Lambda la quale, come vedremo nella Sezione~\ref{sec:phases_of_development}, eseguirà il riconoscimento facciale
della foto appena caricata nel database confontandola con i volti presenti nella collezione di volti di Amazon Rekognition(codice illustrato nel 
Listato~\ref{lambda_code}), quindi, se la foto appena caricata ha una somiglianaza ``threshold'' dell'80\% almeno, si aggiunge al Metadata della foto appena caricata 
il nome, seguito dal cognome, corrispondenti al volto cui la foto è stata associata, se invece non corrisponde a nessun volto, sarà sufficente lasciare il campo Metadata vuoto.

Ritornando all'esecuzione del metodo \textit{send\_email()}, si procede verificando che il campo Metadata sia vuoto o meno, a questo punto, con un controllo 
condizionale If-Else si gestiscono i due casi:
\begin{enumerate}
    \item \textbf{Campo Metadata vuoto}: in questo caso il volto non è stato riconosciuto, succesivamente si darà la possibilità all'utente di memorizzare il volto
    aggiungendolo ai database, in modo da essere riconosciuto successivamente, in questo caso si invia all'utente un e-mail, la quale informa l'utente della ricezione
    di nuova posta, precisando che il volto non è presente nel database e quindi allegando la foto scattata, all'e-mail inviata.
    \item \textbf{Campo Metadata contiene una stringa}: in questo caso il volto è stato riconosciuto, quindi la funzione Lambda ha inserito il nome corrispondente nel
    campo Metadata, a questo punto si può procedere inviando un e-mail che notifica la ricezione di nuova posta, questa volta però non viene allegata la foto ma 
    viene semplicemente indicato il nome del mittente, ottenuto dal campo Metadata che contiene il nome corrispondente.
\end{enumerate}
Infine si procede eliminando, sia localmente su Rapsberry Pi che in Amazon S3, la foto appena processata.

Come è possibile vedere nel codice mostrato nel Listato~\ref{ciclo_while}, se non viene rilevato nessun passaggio di oggetti si ricade nell'Else, qui viene gestito il
caso in cui non viene ricevuta posta per un determinato periodo di tempo.
In questo caso si effettua una sleep di 0.5 secondi, quindi si incrementa di 1 la variabile ``\texttt{contatore}'', a questo punto si averanno solo due casi possibili:
\begin{enumerate}
    \item \textbf{If:} in questo caso viene verificato se la variabile contiene il numero 120, il che significa che è passato 1 minuto, a questo punto si può eseguire il metodo
    \textit{read\_email\_from\_gmail()}, il cui funzionamento verrà esposto più avanti, quindi si azzera la avariabile ``\texttt{contatore}'', in modo da far ripartire
    il conteggio del tempo trascorso.
    \item in questo caso non entro nell'If in quanto la variabile non contiene il numero 120, non essendo  passato ancora un minuto si può procedere semplicemente senza
    far niente, quindi si ritorna all'esecuzione della riga 7 nel Listato~\ref{ciclo_while}.
\end{enumerate}
Come detto precedentemente, viene data all'utente la possibilità di aggiungere volti ai database per renderne possibile il riconoscimento facciale successivamente 
tramite il servizio AWS Amazon Rekognition. Si è pensato di aggiungere volti al database tramite l'inoltro dell'e-mail (quella precedentemente ricevuta che notifica 
la ricezione di posta ), aggiungendo al corpo del messaggio la parola chiave ``ADD'', seguita dal nome e cognome (separati da uno spazio bianco) con cui lo si vuole memorizzare, così facendo si 
aggiungerà quindi il volto presente nell'allegato dell'email associandone le generalità specificate.

Per far ciò verrà usata la funzione \textit{read\_email\_from\_gmail()} la quale viene eseguita ogni minuto. Alla sua esecuzione vengono create le opportune 
connessioni ai server Gmail e quindi si selezionano tutte le e-mail in ricezione, ricevute dall'indirizzo precedentemente impostato, con presente nel corpo del 
messaggio la dicitura ``\texttt{Forwarded message}'' il che significa che saranno selezionate solo le e-mail inoltrate, in modo da fare una prima selezione su quali analizzare.
\begin{lstlisting}[caption=Espressione regolare,captionpos=t,label=regular_expression]
full_name = (re.search(r"([A-Z]{1}[a-z]+\s{1}){2,3}",str(emails_info[num]['Body'])).group()).strip()
\end{lstlisting}
A questo punto si può procedere filtrando ulteriormente le e-mail ricevute analizzando solo quelle che presentano nel corpo del messaggio la parola chiave ``ADD'',
le informazioni di ogni e-mail (oggetto, data, e-mail mittente) vengono memorizzate nella variabile di tipo dictonary ``\textit{emails\_info}'', mentre le immagini in allegato 
vengono memorizzate sulla scheda Raspberry Pi con il nome ottenuto dal corpo del messaggio, ottenuto dall'espressione regolare nel Listato~\ref{regular_expression}
seguito dall'estensione ``.jpg''.
\begin{lstlisting}[language=Python,frame=single,caption=Codice che gestisce la ripetizione dei nomi,captionpos=t,label=repetition_image]
    while True:
        try:
            #se non viene lanciata alcuna eccezione, il file esiste 
            #attualmente, quindi devo modificarlo aggiungendo un 
            #numero dopo il nome
            fp = open("attachment/uploaded/"+file_name, 'rb')
            pass
        except Exception:
            # altrimenti viene lanciata l'eccezione(non esiste un 
            #file con il nome specificato) quindi creo il file
            emails_info[num]['File_Name']= file_name
            emails_info[num]['Full_Name']= full_name
            save_attachment(msg,file_name)
            break
        repetition = repetition + 1 
        file_name = file_name.split(".")[0]+str(repetition)+".jpg"
\end{lstlisting}
Per evitare il problema del nome del file già esistente, come è possibile vedere dal codice illustrato nel Listato~\ref{repetition_image}, viene aggiunto un numero 
alla fine del nome, tutte le foto già caricate invece, si troveranno al percorso ``Desktop/attachment/uploaded/''.

Successivamente si procede inserendo tutte le foto presenti nel percorso specificato (nella scheda Raspberry Pi) in Amazon S3 e Amazon DynamoDB, quindi, solo dopo 
averle caricate è possibile eliminare le e-mail appena processate (come mostrato nel Listato~\ref{delete_email}) in modo da evitare l'aggiunta degli stessi volti alla 
prossima esecuzione della funzione \textit{read\_email\_from\_gmail()}.
\begin{lstlisting}[language=Python,frame=single,caption=Codice che elimina le e-mail,captionpos=t,label=delete_email]
    for msg in id_list: 
        # elimino le email appena processate in quanto le foto sono 
        #state aggiunte ai database
        mail.store(msg, '+FLAGS', '\\Deleted') 
    mail.expunge()
\end{lstlisting}
Infine prima di terminare l'esecuzione, viene chiuso ed effettuato il logout dal server, pronto alla prossima esecuzione della funzione.

\section{Fasi di sviluppo}
\label{sec:phases_of_development}
Prima della scrittura dello script Python, è stato necessario preparare tutto l'occorrente essenziale al funzionamento del progetto, sia lato hardware, come il 
collegamento del sensore e della fotocamera alla scheda Raspberry Pi, che lato software, come l'installazione del sistema operativo Raspbian, l'aggiornamento ed il 
download dei vari pacchetti illustrati nella sezione~\ref{sec:library}, l'interazione con i vari servizi AWS ecc\dots

In questa sezione quindi si vedrà più nel dettaglio la parte appena descritta.

\subsection{Configurazione Raspberry Pi}
All'acquisto della scheda Raspberry Pi viene fornita una scheda SD con preinstallato il sistema operativo Raspbian, una volta installato seguendo varie guide sul web
in cui viene descritta passo passo ogni fase da seguire, si può procedere installando Python con il seguente comando eseguito da terminale.
\begin{lstlisting}[frame=lines]
    sudo apt-get install python-serial
\end{lstlisting}
Per accedere ai servizi AWS è stato installato Boto3, l'SDK di AWS per Python(\ref{sec:boto3}), mentre per semplificare la gestione dei servizi AWS è stata installata
AWS Command Line Interface(AWS CLI), la quale permette di creare qualsiasi oggetto AWS dalla riga di comando senza utilizzare l'interfaccia GUI.

Per la loro installazione sono state eseguite da terminale le righe di comando illustrate di seguito.
\begin{lstlisting}[frame=lines]
    pip install boto3
    pip install awscli
\end{lstlisting} 
Prima di poter usare AWS CLI bisogna però seguire le istruzioni dell'articolo\cite{cli}, infatti dopo aver scelto la regione in AWS(consigliabile il più vicino 
possibile all'area geografica dove si risiede) si può procedere eseguendo il comando ``\textit{aws configure}'', se eseguite correttamente le istruzioni, sarà quindi
possibile utilizzare AWS CLI per qualsiasi operazione sui servizi AWS.

\subsection{Servizi Amazon Web Services}
Nella Sezione~\ref{sec:aws}, abbiamo visto che sono stati utilizzati vari servizi AWS, l'interazione con essi verrà trattata qui di seguito.

Come visto precedentemente Amazon Rekognition non memorizza copie delle immagini analizzate. Al contrario, memorizza i vettori delle caratteristiche del volto come la
rappresentazione matematica di un volto all'interno della raccolta. 
Quindi prima di tutto è stata creata la collection, avendo installato e configurato l'interfaccia della riga di comando di AWS (AWS CLI), è stato utilizzato il 
seguente comando per la sua creazione.
\begin{lstlisting}[frame=lines]
    aws rekognition create-collection --collection-id collection --region eu-central-1 
\end{lstlisting}
Con questo comando verra creata la collection a cui è stato dato il nome ``collection'', specificando che la regione scelta è ``eu-central-1'', in quanto questo server 
è il più vicino all'area geografica dove si risiede. 
A questo punto, si può procedere creando un ruolo dalla console IAM chiamato ``\textsl{lambda\_role}'', come visto nella Sezione \ref{sec:iam}, a cui sono stati associati 
i seguenti permessi:
\begin{itemize}
    \item AmazonRekognitionFullAccess
    \item AmazonDynamoDBFullAccess
    \item AmazonS3FullAccess
    \item IAMFullAccess
\end{itemize}
Così facendo si avranno tutti i permessi per creare e maipolare i vari servizi utilizzati.
Successivamente, viene creata una tabella Amazon DynamoDB. 

In questo progetto la tabella DynamoDB verrà usata come un semplice archivio contenente delle coppie \texttt{chiave-valore},
per mantenere un riferimento di FaceId restituito da Amazon Rekognition e il nome completo della persona. Per la sua creazione è stato eseguito il 
seguente comando.
\begin{lstlisting}[frame=lines]
    aws dynamodb create-table --table-name sender_id \
    --attribute-definitions AttributeName=RekognitionId,AttributeType=S \
    --key-schema AttributeName=RekognitionId,KeyType=HASH \
    --provisioned-throughput ReadCapacityUnits=1,WriteCapacityUnits=1 \
    --region eu-central-1
\end{lstlisting}
Questo comando creerà quindi una tabella DynamoDB chiamata ``\textsl{sender\_id}'' contenete due campi:
\begin{itemize}
    \item \textbf{RekognitionId}
    \item \textbf{FullName}
\end{itemize}
Il primo, ``\textsl{RekognitionId}'', conterrà tutti gli ID univoci ottenuti dall'associazione ai volti della loro rappresentazione matematica tramite Amazon Rekognition, 
mentre il secondo, ``\textsl{FullName}'', conterrà una stringa che rappresenta il nome seguito dal cognome di tutti i volti presenti nella tabella.

Per la creazione del database Amazon S3 è stato usato il seguente comando.
\begin{lstlisting}[frame=lines]
    aws s3 mb s3://images-sender --region eu-central-1
\end{lstlisting}
Questo database è stato creato assegnadogli il nome ``images-sender'', specificando come di consuetudine la regione ``eu-central-1''.
Come ultimo passo, è stata creata la funzione Lambda che viene attivata ogni volta che una nuova foto viene caricata su Amazon S3. Per creare la funzione è stata 
utilizzata la console di AWS Lambda selezionando l'opzione autore da zero (Author from scratch).
Alla funzione Lambda è stato assegnato il nome ``\textsl{photo\_rekognition}'', prima di essere operativa naturalmente si è scritto il codice mostrato nel 
Listato~\ref{lambda_code},
\begin{lstlisting}[language=Python,frame=single,caption=Codice Lambda,captionpos=t,label=lambda_code]
    def lambda_handler(event, context):
        try:
            #recupero il nome del bucket
            bucket = event['Records'][0]['s3']['bucket']['name']
            collectionId='collection'
            #recupero la chiave dell'oggetto caricato in s3
            fileName = urllib.parse.unquote_plus(event['Records'][
            0]['s3']['object']['key'], encoding='utf-8', errors='')
            # verranno selezionati solo i volti con una 
            #somiglianaza dell'80% almeno
            threshold = 80
            rekognition=boto3.client('rekognition')
            dynamodb = boto3.client('dynamodb')
            s3 = boto3.client('s3')
            response=rekognition.search_faces_by_image(
                        CollectionId=collectionId,
                        Image={'S3Object':{'Bucket':bucket,
                        'Name':fileName}},
                        FaceMatchThreshold=threshold,
                        MaxFaces=maxFaces)
            faceMatches=response['FaceMatches']
            if faceMatches:
                # recupero il nome(grazie all'id del volto 
                #riconosciuto)
                face = dynamodb.get_item(
                    TableName='sender_id',  
                    Key={'RekognitionId': {'S': faceMatches[0]
                    ['Face']['FaceId']}}
                    )
                if 'Item' in face:
                    name = face['Item']['FullName']['S']
                    s3.copy_object(
                        Bucket=bucket,Key=fileName,
                        CopySource={"Bucket": bucket, 
                        "Key": fileName},
                        Metadata={"x-amz-meta-fullname": name},
                        MetadataDirective="REPLACE")
                else:
                    print ('no match found in person lookup')
            else :
                print("No matching face")
        except Exception as e:
            print("Funzione terminata con errore.....")
            print
            print(str(e))
\end{lstlisting}
succesivamente è stato configurando l'evento che ne farà lanciare l'esecuzione, in particolare si tratta di un evento di tipo ``\textsl{ObjectCreatedByPut}'', 
inoltre è stato specificato il prefisso ``\textsl{image}'' la cui funzione verrà esposta di seguito.

Ricapitolando, ogni volta in cui viene creato un oggetto tramite il suo inserimento, assegnado come nome del file il prefisso ``\textsl{image}'' seguito dall'estensione del 
file, sarà eseguita la funzione Lambda che confronta il volto appena caricato nel database Amazon S3 con quelli presenti nella collection di Amazon Rekognition, se 
qualcuno di essi ha una somiglianza di almeno l'80\% , si procederà inserendo nei Metadata del volto caricato in Amazon S3. il nome e cognome (separati da uno spazio bianco)
corrispondenti al volto riconosciuto, ricavati da Amazon DynamoDB, tramite il corrispondente ID.

\subsection{PuTTY}
Come abbiamo visto nelle altre Sezioni, il codice viene eseguito sulla scheda Raspberry Pi, in Figura ~\ref{photo_raspberry}, per la scrittura dello script Python, 
l'assemblaggio dei vari componenenti e per eventuali debug si è optato per l'utilizzo di PuTTY, un software open source utilizzato per la gestione in remoto dei server 
usando SSH, il cui sviluppo risale ai primi mesi del 1999 da parte del programmatore Simon Tatham.

Questo software per Windows, rende possibile l’interazione con sistemi Unix remoti grazie all’utilizzo dei protocolli SSH, Rlogin e Telnet. Per il progetto è stato
utilizzato il protocollo SSH il quale, mantenendo una connessione Internet attiva, permette di stabilire una sessione cifrata remota con un'altra macchina 
nella rete (Raspberry Pi nel nostro caso).

Una volta connessi al server remoto, ogni tipologia di comando digitato attraverso la piattaforma PuTTY arriverà al server remoto con cui si è appena effettuata
la connesione, in egual modo, ogni risposta ai comandi riprodotta dal server verrà trasferita sulla piattaforma locale personale. Così facendo sarà possbile 
utilizzare la nostra scheda Raspberry per tutte le operazioni necessarie alla realizzazione del progetto.

\subsection{WinSCP}
Per la creazione o per tutte le manipolazioni o trasferimenti di file è stato usato Winscp, un client grafico open source per Windows per SFTP e FTP la cui funzione
principale è quella di copiare in modo sicuro file tra un computer locale(macchina personale Windows) e uno remoto(scheda Raspberry Pi).

Utilizzando questa applicazione si è trasferito facilmente qualsiasi file dalla macchina Windows locale al Rapsberry Pi, inoltre si è semplificata la manipolazione dei 
file in quanto viene messa a disposizione un'interfaccia grafica molto semplice e intuitiva.

\section{Testing}
\label{ch:testing}
In questa sezione verranno illustrati i vari test eseguiti sul codice.

Prima di tutto è stato scritto un piccolo script bash(Listato~\ref{bash_code})il quale, tramite crontab, ad ogni avvio della scheda Raspberry Pi, ed ogni 5 minuti, esegue 
il codice del file chiamato ip.txt nel quale viene memorizzato l’IP del Raspberry, lo script bash memorizza nella variabile ``\textsl{IPAddr}'' l'indirizzo IP attuale,
dopodiché con un ciclo while legge il file e confronta l’IP attuale con quello precedentemente memorizzato nel file che si sta leggendo, se è diverso significa che il 
Raspberry ha cambiato IP(non è più quello memorizzato nel file) quindi sovrascrive l’indirizzo IP memorizzato precedentemente con quello attuale, quindi manda un e-mail all' 
indirizzo precedentemente impostato con l’IP attuale, altrimenti non fa niente.
\begin{lstlisting}[language=Python,frame=single,caption=Script bash,captionpos=t,label=bash_code]
    import socket
    # memorizzo l'hostname nella variabile     
    hostname = socket.gethostname()
    # memorizzo nella variabile l'indirizzo IP dell' "hostname"              
    IPAddr = socket.gethostbyname(hostname)     
    f= open("/home/pi/Desktop/ip.txt", "r")
    ipSaved = f.readline()
    # se l'IP salvato nel file "ip.txt" non corrispinde all'indirizzo IP memorizzato nella variabile, modifico il file e mando l'email
    if ipSaved != IPAddr:
        f = open("/home/pi/Desktop/ip.txt","w")
        f.write(IPAddr)
        # funzione che invia l'email con l'indirizzo IP attuale
        sendEmail(IPAddr)
\end{lstlisting}
Dopodiché è stato testato il funzionamento del sensore di movimento eseguendo il codice mostrato nel Listato~\ref{sensor_code} il quale, dopo aver importato le 
opportune librerie, testa la connessione al sensore, se va a buon fine verrà semplicemente stampato a video la prossimità dell'oggetto rilevato.
\begin{lstlisting}[language=Python,frame=single,caption=Codice sensore,captionpos=t,label=sensor_code]
#!/usr/bin/python3
import qwiic_proximity
import time
import sys
print("\nSparkFun Proximity Sensor VCN4040 Example 1\n")
oProx = qwiic_proximity.QwiicProximity()
if oProx.isConnected() == False:
    print("The Qwiic Proximity device isn't connected to the system. Please check your connection", \
        file=sys.stderr)
    return
oProx.begin()

while True:
    proxValue = oProx.getProximity()
    print("Proximity Value: %d" % proxValue)
    time.sleep(.4)
\end{lstlisting}
Per testare il funzionamento della fotocamera è stato eseguito il codice mostrato nel Listato~\ref{code_camera} il quale, dopo aver importato le opportune librerie 
esegue una sleep di 5 secondi(in modo da dare il tempo ai sensori di luce della fotocamera di stabilizzarsi)scatta una foto e la memorizza nel Desktop .
\begin{lstlisting}[language=Python,frame=single,caption=Codice Python che scatta la foto,captionpos=t,label=code_camera]
#!/usr/bin/python3
from picamera import PiCamera
from time import sleep
camera = PiCamera()
camera.start_preview()
#attendo 5 secondi per dare il tempo ai sensori di luce della  
#fotocamera di stabilizzarsi, quindi scatto la foto con il comando
#nella riga 10
sleep(5)
camera.capture('/home/pi/Desktop/image.jpg')
camera.stop_preview()
\end{lstlisting}
Appurato il funzionamento dei due componenti, si può proseguire creando i vari database su AWS come mostrato nella Sezione~\ref{sec:phases_of_development}. 
Con il codice a seguire sono state inserite alcune immagini nel database Amazon S3
\begin{lstlisting}[language=Python,frame=single]
#!/usr/bin/python
import boto3
s3 = boto3.resource('s3')
# Get list of objects for indexing
images=[('aldo.jpg','Aldo Bushaj'),
      ('nino.jpg','Antonino Bushaj'),
      ('silvestro.jpg','Silvestro Frisullo'),
      ('vlad.jpg','Vlad Axinte'),
      ('florian.jpg','Florian Karici'),
      ('supermario.jpg','Super Mario')
      ]

# Iterate through list to upload objects to S3   
for image in images:
    file = open(image[0],'rb')
    ret = s3.Bucket("images-sender").put_object(Key=image[0], Body=file, Metadata={'FullName':image[1]})
\end{lstlisting}
specificando per ognuno, nei rispettivi Metadata, il nome seguito dal cognome.
A questo punto, con il codice nel Listato~\ref{index_code}, si procede inserendo i volti presenti in Amazon S3 prima nella collection di Amazon Rekognition, con il codice

\begin{lstlisting}[language=Python,frame=single,caption=Codice inserimento volti in DynamoDB e collection,captionpos=t,label=index_code]
def update_index(tableName,faceId, fullName):
    response = dynamodb.put_item(
    TableName=tableName,
    Item={
        'RekognitionId': {'S': faceId},
        'FullName': {'S': fullName}
        }
    )
def index_faces(bucket, key, collection_id, attributes=(), region="eu-central-1"):
    # indica se tutte le immagini sono state caricate correttamente su dynamodb
    status = True 
    for index in range(0,len(key)):
		rekognition = boto3.client("rekognition", region)
		response = rekognition.index_faces(
			Image={
				"S3Object": {
					"Bucket": bucket,
					"Name": key[index],
				}
			},
			CollectionId=collection_id,
			ExternalImageId=key[index],
			DetectionAttributes=attributes,
		)
		if response['ResponseMetadata']['HTTPStatusCode'] == 200:
			faceId = response['FaceRecords'][0]['Face']['FaceId']
		
			# head_object ritorna esclusivamente i matadati dell'oggetto indicato all'interno del bucket specificato
			ret = s3.head_object(Bucket=bucket,Key=key[index])
			personFullName = ret['Metadata']['fullname']
			print("Uploaded in collection : "+personFullName)
   
			update_index('sender_id',faceId,personFullName)
		else:
			status = False
		
    return status

if index_faces(BUCKET, all_key, COLLECTION):
	print("Images succesfully uploaded")
else: 
    print("ERRORR.... Images not uploaded")

\end{lstlisting}
della funzione ``\textsl{index\_faces}'', il quale una volta recuperata l'immagine del volto precedentemente caricata nel database Amazon S3, con la funzione nativa di 
Amazon Rekognition ``\textsl{rekognition.index\_faces}'', per ciascun volto, viene calcolato l'ID corrispondente e quindi memorizzato nella collection.
Succesivamente, si aggiunge il volto appena inserito nella collection anche nella tabella DynamoDB, per far ciò viene usato il metodo 
``\textsl{update\_index}'' il quale una volta eseguito, inserisce nella tabella il corrispondente ID volto nel campo RekognitionId e il nome seguito dal cognome,
(recuperati dal Metadata della foto) nel campo FullName. 

Verificato il funzionamento del codice appena descritto, non rimane altro che addattare le due funzioni ad un'esecuzione automatica dei metodi, ogni qualvolta viene
ricevuta nuova posta.